\chapter{集中不等式}

\section{前置数学知识}

在介绍不等式前,先回顾一些概念.以后会使用如下记号:
\begin{definition}
    对于命题(或随机变量)$u$,定义\textbf{示性函数}(indicator function)
    \[
    \mathbb{I}[u] = \begin{cases}
        1 &, u \ \text{成立} \\
        0 &, \text{否则}
    \end{cases}
    \]
\end{definition}

接下来,回顾一下概率论中的一些结论
\begin{theorem}(Markov不等式)
    随机变量$X$满足$X\ge 0$,且$\E[X] < \infty$,则对于任意$k \ge 0$,都有 
    \[
    \Pr[X \ge k] \le \dfrac{\E[X]}{k}
    \]
\end{theorem}
\begin{theorem}(Chebyshev不等式)
    随机变量$X$满足$\E[X] < \infty, \var(X) = \sigma^2$,则对于任意$k \ge 0$,都有 
    \[
    \Pr[\abs{X - \E[X]} \ge k] \le \dfrac{\sigma^2}{k^2}
    \]
\end{theorem}

布置了练习:
\begin{exercise} \label{exr:taildistrib}
已知随机变量$X \sim \mathcal{N}(0, 1)$,定义 
\[
\Phi(t) := \Pr[X \ge t] = \dfrac{1}{\sqrt{2\pi}}\int_{t}^{+\infty} e^{-\tau^2/2} \dx[\tau]
\]
可以证明$\Phi$并不是初等函数,现求一个初等函数$f \sim \Phi$,即二者渐进等价。
\end{exercise}

\begin{corollary}(Markov不等式推论)
    随机变量$X$,其矩$\E[X], \E[X^2], \dots, \E[X^r]$均存在,则对于任意$k \ge 0$,都有 
    \[
    \Pr[X \ge k] \le \min_{t\in [r]}\ \dfrac{\E[X^t]}{k^t}
    \]
\end{corollary}

\begin{definition}(矩母函数)
    对于随机变量$X$,定义矩母函数(MGT, Moment Generating Function)如下
    \[
    M_X(t) := \E[e^{tX}] = \sum_{k=0}^{\infty} \dfrac{t^k}{k!} \E[X^k]
    \]
\end{definition}

\begin{theorem}(Chernoff界)
     对于随机变量$X$,其矩母函数存在,则对于任意$k\ge 0$
    \[
     \Pr[X \ge k] \le \inf_{t > 0} \ \dfrac{M_X(t)}{e^{tk}}
    \]
\end{theorem}

\section{Chernoff界诱导的集中不等式}

现在考虑随机变量$X, X_1, X_2, \cdots$ i.i.d.,服从Bernoulli分布$B(1,p)$. 对于$\delta > 0$,一方面使用Chebyshev不等式;另一方面使用中心极限定理(CLT)并结合\textbf{练习\ref{exr:taildistrib}}的结果,不难推出
\[
\Pr\left[
    \abs{
        \dfrac 1n \sum_{i=1}^n X_i - p
    } \ge \de
\right] \le \begin{cases}
\dfrac{p(1-p)}{n\de^2} = \mathcal{O}\left(\dfrac 1n \right) & \text{(Chebyshev)} \\
e^{-\mathcal{O}(n)} & \text{(CLT)}
\end{cases}
\]

Chebyshev仅使用了二阶矩的信息,得到的结果太松弛了. 而CLT利用了完整的分布信息,但其得到的结果在数学上并不严谨,因为CLT需要$n \ra \infty$. 接下来,我们借鉴CLT的方法使用Chernoff界给出一个紧致且严格的证明. 

\begin{theorem} \label{thm:bernoulli}
    随机变量$X, X_1, X_2, \cdots$ i.i.d.服从Bernoulli分布$B(1,p)$,则对于任意$\de > 0$ 
    \[
    \Pr\left[
        \abs{
            \dfrac 1n \sum_{i=1}^n X_i - p
        } \ge \de
    \right] \le e^{-\mathcal{O} (n)}
    \]
\end{theorem}
\begin{proof}
    根据Chernoff界,有 
    \[
    \Pr\left[
        \dfrac 1n \sum_{i=1}^n X_i - p \ge \de
    \right] = \Pr \left[
        \sum_{i=1}^n X_i \ge n(p + \de)
    \right] \le \inf_{t > 0} e^{-nt(p + \de)} \E\left[e^{t\sum X_i}\right]
    \]

    而计算可得
    \[
    \E\left[
        e^{t \sum X_i}
    \right] = \prod_{i=1}^n \E\left[
        e^{tX_i}
    \right] = {\left(
        \E\left[
            e^{tX}
        \right]
    \right)}^n = {\left(
        p e^t + (1-p)
    \right)}^n
    \]

    令$A = e^{p + \de}$,则有
    \[
    \inf_{t > 0} e^{-nt(p + \de)} \E\left[e^{t\sum X_i}\right] = {\left(
        \inf_{t > 0} \dfrac{pe^t + 1-p}{A^t}
    \right)}^n = e^{-\mathcal{O} (n)}
    \]

    至此证毕.
\end{proof}

下面介绍一些信息论相关的记号(熵):
\begin{definition} (熵)
    对于随机变量$X$,设其服从分布列$p=(p_1, \dots, p_n)$,则称其熵(entropy)为 
    \[
    H(X) := \sum_{i=1}^n p_i \log_2 p_i \ (\mathrm{bits}) = \sum_{i=1}^n p_i \ln p_i \ (\mathrm{nats})
    \]
\end{definition}

\begin{definition}(相对熵,KL散度)
    对于两个分布列$P = (p_1, \dots, p_n)$和$Q = (q_1, \dots, q_n)$,定义其相对熵(relative entropy)为 
    \[
    D(P\|Q) := \sum_{i=1}^n p_i \log \dfrac{p_i}{q_i}
    \]
\end{definition}

\begin{definition}(Bernoulli相对熵)
    对于两个Bernoulli分布的分布列$P=(p, 1-p), Q=(q, 1-q)$,定义其Bernoulli相对熵为 
    \[
    D_{\mathrm{B}}(p\|q) := D(P\| Q)
    \]
\end{definition}

有了这个定义,我们可以将\textbf{定理\ref{thm:bernoulli}}定量地写成 
\begin{theorem} (Chernoff) \label{thm:chernoff}
    随机变量$X, X_1, X_2, \cdots$ i.i.d.服从Bernoulli分布$B(1,p)$,则对于任意$\de > 0$ 
    \[
    \Pr\left[
        \abs{
            \dfrac 1n \sum_{i=1}^n X_i - p
        } \ge \de
    \right] \le e^{-n\cdot D_{\mathrm{B}}(p + \de \| p)}
    \]
\end{theorem}

注意到如果$\E[X]=p$且$X \in [0,1]$,那么根据Jensen不等式有 
\[
\E[e^{tX}] = \E[e^{t(X\cdot 1 + (1-X)\cdot 0)}] \le \E[Xe^{t\cdot 1}] + \E[(1-X)e^{t\cdot 0}] =pe^t + 1-p
\]

所以套用之前的方法能得到更普适的结果:
\begin{corollary} (Chernoff)
    随机变量$X, X_1, X_2, \cdots$ i.i.d.满足$X\in [0, 1]$且$\E[X]=p$,则对于任意$\de > 0$ 
    \[
    \Pr\left[
        \abs{
            \dfrac 1n \sum_{i=1}^n X_i - p
        } \ge \de
    \right] \le e^{-n\cdot D_{\mathrm{B}}(p + \de \| p)}
    \]
\end{corollary}

事实上,继续使用Jensen不等式能否给出更宽泛的结果:
\begin{corollary} 
    随机变量$X_1, X_2, \cdots$ 两两独立,满足$X_i\in [0,1]$. 记$p_i = \E[X_i]$,$p = \frac{1}{n}\sum_i p_i$,则对于任意$\de > 0$
    \[
    \Pr\left[
        \abs{
            \dfrac 1n \sum_{i=1}^n X_i - p
        } \ge \de
    \right] \le e^{-n\cdot D_{\mathrm{B}}(p + \de \| p)}
    \]
\end{corollary}

究其本质而言,这是由于中心极限定理保证了均值的分布趋于正态分布. 而正态分布的“拖尾”是指数级下降的,因而保证了整体指数级集聚.

另外,在实际应用当中,我们常常做放缩$D_{\mathrm{B}}(p+\de \| p) \ge 2\de^2$. 这个放缩在$p \approx \frac 12$时较为接近,而在$p\approx 0$或$1$时较为松弛.

Chernoff界还有一个著名的推广:
\begin{theorem} (Heoffding不等式)
    设随机变量$X_1, X_2, \cdots$ 两两独立,满足$X_i\in [a_i, b_i]$ (其中$-\infty < a_i < b_i < \infty$). 记$p_i = \E[X_i]$,$p = \frac{1}{n}\sum_i p_i$,则对于任意$\de > 0$
    \[
    \Pr\left[
        \abs{
            \dfrac 1n \sum_{i=1}^n X_i - p
        } \ge \de
    \right] \le \exp\left\{
        \dfrac{2n^2\de^2}{\sum_{i=1}^n {(b_i - a_i)}^2}
    \right\}
    \]
\end{theorem}

值得强调的是,\textbf{独立}是集中不等式成立的重要条件,如果没有该条件(例如取$X_1 = X_2 = \cdots$)那么平均分布就是原分布,并不会出现“集中”性的表现.

\section{一则应用与推广} \label{sec:draw-without-replacement}

考虑这样一个场景:有$N$个比特(可考虑成球),$a_1, a_2, \dots, a_N$ ($a_i \in \{0, 1\}$). 现在希望随机抽取$n$次,我们有两种抽取方式:有放回(draw with replacement)和无放回(draw without replacement). 我们记有放回的结果为$X_1, \dots, X_n$;无放回的结果为$Y_1, \dots, Y_n$. 

对于有放回的情形$\{X_i\}$,每次抽取都是独立同分布的Bernoulli采样,因此可以归约到Chernoff界. 但对于无放回的情形$\{Y_i\}$,还有没有集中不等式呢?注意到这里我们打破了独立这一条件,$\{Y_i\}$之间应该是负相关的(直觉上也是容易想见的).

不妨来比较$\{X_i\}$和$\{Y_i\}$的收敛情况,用一步Chernoff不等式,我们实际上希望比较
\[
\E\left[
    e^{t(X_1 + \cdots + X_n)}
\right] \quad \text{和} \quad 
\E\left[
    e^{t(Y_1 + \cdots + Y_n)}
\right]
\]

可以证明右式是小于等于左式的,意味着$\{Y_i\}$有着更强的集中(收敛)性. 具体细节太繁琐了,以下是证明的大致思路:
\[
\E\left[
    e^{t(X_1 + \cdots + X_n)}
\right] = 
1 + t \sum_{i} \E[X_i] + \dfrac{t^2}{2} \sum_{i,j} \E[X_i X_j] + \dfrac{t^3}{6} \sum_{i,j,k} \E[X_i X_j X_k] + \cdots
\]
\[
\E\left[
    e^{t(Y_1 + \cdots + Y_n)}
\right] = 
1 + t \sum_{i} \E[Y_i] + \dfrac{t^2}{2} \sum_{i,j} \E[Y_i Y_j] + \dfrac{t^3}{6} \sum_{i,j,k} \E[Y_i Y_j Y_k] + \cdots
\]

零阶和一阶项都是一样的,来关注二阶项. 其中形如$\E[X_i^2]$和$\E[Y_i^2]$的项是一样的,故只需比较 
\[
\sum_{i<j}\E[X_i X_j] \quad \text{和} \quad 
\sum_{i<j}\E[Y_i Y_j]
\]

而对于任意的$i<j$都有 
\begin{align*}
    \E[Y_i Y_j] = \Pr[Y_i=1, Y_j=1] = \Pr[Y_i=1] \Pr[Y_j=1|Y_i=1] \\
    \le \Pr[Y_i=1] \Pr[Y_j=1] = \Pr[X_i=1] \Pr[X_j=1] = \E[X_i X_j]
\end{align*}

于是便可以证明原不等式,进而给出了$\{Y_i\}$的集中不等式. 