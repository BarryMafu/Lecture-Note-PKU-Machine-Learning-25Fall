\chapter{作业题目及解答} \label{chap:homework}

% ----- HW 1 -----

\begin{exercise} 
已知随机变量$X \sim \mathcal{N}(0, 1)$,定义 
\[
\Phi(t) := \Pr[X \ge t] = \dfrac{1}{\sqrt{2\pi}}\int_{t}^{+\infty} e^{-\tau^2/2} \dx[\tau]
\]
可以证明$\Phi$并不是初等函数,现求一个初等函数$f \sim \Phi$,即二者渐进等价。
\end{exercise}

\textbf{分析} \quad 先分析一下这个问题:显然$\Phi(t) \ra 0$,所以要$f \ra 0$. 现在$\Phi$的形式非初等不便于分析,不妨考虑$\Phi'(t) = \varphi(t)$,为此可以运用L'Hospital法则:
\[
\lim_{t \ra +\infty} \dfrac{\Phi(t)}{f(t)} = \lim_{t \ra +\infty} \dfrac{\Phi'(t)}{f'(t)} = \lim_{t \ra +\infty} \dfrac{-C e^{-t^2/2}}{f'(t)}
\]
其中$C$是某常数. 我们希望$f \sim \Phi$,也就是上述极限为1, 所以为了化简,我们希望$f'(t)$中也出现$e^{-t^2/2}$的形式. 回忆到 
\[
v(x)e^{u(x)} = \Big(v'(x) + u'(x)v(x) \Big) \cdot e^{u(x)}
\]
因此我们不妨设$f(t)$形如$g(t) e^{-t^2/2}$,此时
\[
\lim_{t \ra +\infty} \dfrac{-C e^{-t^2/2}}{f'(t)}
= \lim_{t \ra +\infty} \dfrac{-C e^{-t^2/2}}{\big[g'(t) - tg(t)\big] \cdot e^{-t^2/2}} = \lim_{t \ra +\infty}\dfrac{-C}{g'(t) - tg(t)}
\]
欲使上式为1,就要
\[
\lim_{t\ra +\infty}  tg(t) - g'(t) = \dfrac{1}{C}
\]
简便起见只考虑$C=1$,之后再给$g$乘上系数. 注意!这里千万不要把其当作$-g'(t) + tg(t) = 1$这样的一阶线性常微分方程求解,因为其解不保证初等. 如果你尝试求解ODE会发现解得$f = \Phi$确实不初等. 在这里,我们只需要考虑到$t \ra +\infty$,所以我们令$g(t) = t^{-1}$即合意.

\begin{solution}
构造初等函数 
\[
f(t) = \dfrac{1}{\sqrt{2\pi}} \cdot \dfrac{e^{-t^2/2}}{t}
\]
根据L'Hospital法则,可以验证
\[
\lim_{t\ra +\infty} \dfrac{\Phi(t)}{f(t)} = \lim_{t \ra +\infty} \dfrac{\Phi'(t)}{f'(t)} = \lim_{t \ra +\infty} \dfrac{-e^{-t^2/2}}{\left(-\dfrac{1}{t^2} - t\cdot \dfrac{1}{t}\right) e^{-t^2/2}} = \lim_{t\ra \infty}\dfrac{1+t^2}{t^2} = 1
\]
因此$f$和$\Phi$渐进等价. 
\end{solution}

事实上,我们上面给出的是 Mills 渐近展开的首项,完整的是:
\[
\Phi(t) \sim \dfrac{1}{\sqrt{2\pi}} \cdot {e^{-\frac{t^2}{2}}}
\cdot \left(
    \dfrac{1}{t} - \dfrac{1}{t^3} + \dfrac{1 \cdot 3}{x^5} - \dfrac{1 \cdot 3 \cdot 5}{x^7} + \cdots
\right)
\]

% ----- HW 2 ------

\begin{exercise}
    记 
    \[
    \ca F = \left\{
        \mathrm{sgn}(\bd{w}^{\top}\cdot \bd{x} + b): \bd{w}\in \R^d, b\in \R 
    \right\}
    \]
    其中$\mathrm{sgn}(\cdot)$是符号函数,定义为 
    \[
    \mathrm{sgn}(z) = \begin{cases}
    1 & ,z > 0\\
    -1 & ,z \le 0
    \end{cases}
    \]

    试证明:$\ca F$的VC维度是$d+1$.
\end{exercise}

\begin{proof}
记$f_{\bd{w}, b}(\bd{x}) = \mathrm{sgn}(\bd{w}^\top \cdot \bd{x} + b) \in \ca F$. 我们分两部分来证明:
\begin{enumerate}
    \item[\Circled{1}] 现取一组$\bd{x}_1, \dots, \bd{x}_{d+1}$使得其可以在$\ca F$下取到任意$d+1$维比特串. 令
    \begin{align*}
        \bd{x}_1 & = {(1, 0, \dots, 0, 0)}^\top \\
        \bd{x}_2 & = {(0, 1, \dots, 0, 0)}^\top \\
                & \vdots \\
        \bd{x}_d & = {(0, 0, \dots, 0, 1)}^\top \\
        \bd{x}_{d+1} & = {(0, 0, \dots, 0, 0)}^\top
    \end{align*}
    
    那么对于任意的$\bd{y} = (y_1, \dots, y_{d+1}) \in \{\pm 1\}^{d+1}$,取 
    \[
    \bd{w}_0 = 2\bd{y}_{1:d}^\top = {(2y_1, \dots, 2y_{d})}^\top, \quad b_0 = y_{d+1}
    \]

    则由于$\bd{y}$的各个元素取值于$\{\pm 1\}$,不难证明
    \begin{align*}
        f_{\bd{w}_0, b_0}(\bd{x}_j) & = \mathrm{sgn}(\bd{w}_0^\top \bd{x}_j+ b_0) = \mathrm{sgn}(2y_j + y_{d+1}) = y_j &, j=1,\dots,d \\
        f_{\bd{w}_0, b_0}(\bd{x}_{d+1}) & = \mathrm{sgn}(\bd{w}_0^\top \bd{0}+ b_0) = \mathrm{sgn}(y_{d+1}) = y_{d+1}
    \end{align*}
    
    至此我们说明对于任意$d+1$维比特串,都存在$f\in \ca F$使得$(\bd{x}_1, \dots, \bd{x}_{d+1})$在$f$的像是该比特串. 

    \item[\Circled{2}] 对于任意$\bd{x}_1, \dots, \bd{x}_{d+2}$,往证明一定有其取不到的比特串. 我们令$\tilde{\bd{x}}_j = {(\bd{x}_j^\top, 1)}^\top \in \R^{d+1}$(即添加一个1)那么根据向量空间基本定理可知这些向量线性相关,即存在一组不全为零的实数$\ld_1, \dots, \ld_{d+2}$使得 
    \[
    \sum_{j=1}^{d+2} \ld_j \cdot \tilde{\bd{x}}_j = \bd{0} \quad \Ra \quad \sum_{j=1}^{d+2} \ld_j \cdot \bd{x}_j = \sum_{j=1}^{d+2} \ld_j = 0
    \]

    现在任取$\bd{w}\in \R^d, b\in \R$,我们知道 
    \begin{equation} \label{eq:hw2-1}
    \sum_{j=1}^{d+1} \ld_j \cdot \big(\bd{w}^{\top}\bd{x}_j + b\big) = b \cdot \sum_{j=1}^{d+1} \ld_j = 0
    \end{equation}

    因此我们可以断言
    \[
    \big(
        f_{\bd{w}, b}(\bd{x}_1), \dots, f_{\bd{w}, b}(\bd{x}_{d+2})
    \big) \neq \big(
        \mathrm{sgn}(\ld_1), \dots, \mathrm{sgn}(\ld_{d+2})
    \big)
    \]

    否则\hyperref[eq:hw2-1]{A.1式}左侧与$\sum_j \abs{\ld_j}$同号,而这是正数$>0$会导致矛盾.
    
    至此我们说明了一定存在取不到的$d+2$维比特串.
    
\end{enumerate}
综上所述,我们证明了$\ca F$的VC维度是$d+1$.
\end{proof}